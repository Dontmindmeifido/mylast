\documentclass[12pt, a4paper]{article}

\usepackage{listings}
\usepackage{xcolor}

% Setup the style
\lstset{
    basicstyle=\ttfamily\small,
    backgroundcolor=\color{white},
    frame=single,
    breaklines=true
}

% --- BASIC SETUP ---
\usepackage[utf8]{inputenc}
\usepackage[T1]{fontenc}
\usepackage{mathptmx}       % Times New Roman
\usepackage{geometry}
\geometry{top=2.54cm, bottom=2.54cm, left=2.54cm, right=2.54cm}

% --- FORMATTING ---
\setlength{\parindent}{0pt}
\setlength{\parskip}{12pt}
\linespread{1.15}

% --- PACKAGES ---
\usepackage{graphicx}
\usepackage{verbatim}       % Used for code blocks
\usepackage{hyperref}       % Links (Default settings, no colorlinks error)

% --- MANUAL ROMANIAN TITLES ---
\renewcommand{\contentsname}{Cuprins}
\renewcommand{\listfigurename}{Listă de figuri}
\renewcommand{\listtablename}{Listă de tabele}
\renewcommand{\refname}{Bibliografie}
\renewcommand{\figurename}{Figură}
\renewcommand{\tablename}{Tabel}

% --- METADATA ---
\title{\textbf{MyLast}\\ \large Documentație}
\author{Gaitur Ruslan}
\date{\today}

\begin{document}

\maketitle
\thispagestyle{empty}
\hrule
\vspace{1em}

\tableofcontents
\newpage

\section{Prezentare Generală}

\textbf{MyLast} este un script Bash conceput pentru a analiza jurnalele de autentificare ale sistemului situate în \texttt{/var/log/}. Acesta reproduce funcționalitatea comenzilor Linux standard \texttt{last} și \texttt{lastb}.

Scriptul ia date din fișierele \texttt{auth.log.x} (inclusiv cele comprimate cu gzip) pentru a reconstrui istoricul sesiunilor utilizatorilor fizici și a identifica încercările de autentificare eșuate.

\section{Caracteristici Principale}
\begin{itemize}
    \item \textbf{Procesare Jurnale:} Citește și concatenează \texttt{auth.log}, \texttt{auth.log.1}, etc.
    \item \textbf{Reconstruirea Sesiunii:} Reconstruiește exact firul de autentificare, deconectare al userului și shutdown al sistemului.
    \item \textbf{Reconstruirea Eșecurilor de Logare:} Determină sesiunile în care au avut loc eșecuri de autentificare.
    \item \textbf{Filtrare:} Permite filtrarea rezultatelor după constrângeri de timp specifice (De la, Până la, Prezent), dar la fel și după numărul de afișări.
\end{itemize}

\section{Instalare și Cerințe}

\subsection{Dependințe}
Asigurați-vă că următoarele sunt instalate:
\begin{itemize}
    \item \texttt{bash} (Mediul Shell)
    \item \texttt{less} (Citirea fișierelor text și, în special, gzip)
    \item \texttt{awk}, \texttt{cut}, \texttt{grep} (Procesare text)
\end{itemize}

\subsection{Permisiuni}
Deoarece scriptul citește din \texttt{/var/log/auth.log}, necesită de obicei privilegii \textbf{root}.

\begin{lstlisting}[language=bash]
sudo ./mylast.sh [...]
\end{lstlisting}

\section{Utilizare}

Scriptul acceptă argumente în linia de comandă pentru a determina modul de operare și logica de filtrare.

\subsection{Sintaxă}

\begin{lstlisting}[language=bash]
./mylast.sh <mod> [flag_filtrare] [valoare_filtrare]
\end{lstlisting}


\subsection{Moduri de Apel \texttt{<mod>}}
Determină modul de analiză:

\begin{table}[h]
\centering
\begin{tabular}{|l|l|}
\hline
\textbf{Flag} & \textbf{Descriere} \\ \hline
\texttt{-l} & \textbf{Ultimele Logări:} Analizează sesiunile reușite. \\ \hline
\texttt{-lb} & \textbf{Logări Eșuate:} Analizează încercările eșuate. \\ \hline
\end{tabular}
\caption{Moduri Principale de Execuție}
\end{table}

\subsection{Opțiuni de Filtrare \texttt{[flag\_filtrare]}}
Acești parametri sunt valabili doar pentru modul \texttt{-l} de rulare al scriptului.

\begin{table}[h]
\centering
\begin{tabular}{|l|l|}
\hline
\textbf{Flag} & \textbf{Format} \\ \hline
\texttt{-n} & Afișează ultimele N intrări. \\ \hline
\texttt{-s} & Arată sesiunile \textbf{de la} data specificată. \\ \hline
\texttt{-t} & Arată sesiunile \textbf{până la} data specificată. \\ \hline
\texttt{-p} & Arată sesiunile \textbf{active} la timpul specificat. \\ \hline
\end{tabular}
\caption{Opțiuni de Filtrare}
\end{table}

\subsection{Parametri de Filtrare \texttt{[valoare\_filtrare]}}
Reprezintă valoarea ce ține de filtrare. Fiecare flag de filtru are un format unic pentru propria valoare.

\begin{table}[h]
\centering
\begin{tabular}{|l|l|l|}
\hline
\textbf{Flag} & \textbf{Format} & \textbf{Descriere} \\ \hline
\texttt{-n} & \textbf{INT} & Număr întreg \\ \hline
\texttt{-s} & \textbf{YYYY-MM-DD-HH-mm} & Data și timp \\ \hline
\texttt{-t} & \textbf{YYYY-MM-DD-HH-mm} & Data și timp \\ \hline
\texttt{-p} & \textbf{YYYY-MM-DD-HH-mm} & Data și timp \\ \hline
\end{tabular}
\caption{Format Acceptat Pentru Parametrii Filtrării}
\end{table}

\subsection{Interfața Vizuală}

Deși \textbf{MyLast} este o aplicație bazată pe linia de comandă, interfața sa vizuală este definită de formatarea strictă a datelor returnate la \textit{Standard Output}. Interacțiunea grafică are loc exclusiv în cadrul terminalului.

\subsubsection{Arhitectura Interfeței}
Scriptul utilizează un \textit{layout tabular bazat pe text} pentru a reprezenta datele. Structura vizuală este proiectată pentru lizibilitate maximă, asemenea comenzilor \texttt{last, lastb}.

Fiecare linie afișată reprezintă o entitate distinctă și este împărțită în coloane vizuale delimitate prin spațiere dinamică.

\subsubsection{Diagrama de Afișare}

\begin{verbatim}
|--------------------------------------------------------------|
|Utilizator|Sesiune|   Data   |Ore de Început și Sfârșit|Durată|
|----------|-------|----------|-------------------------|------|
| student1 |  c2   |2025-12-01|     12:30-14:49         |(2:19)|
| student2 |  c3   |2025-12-01|     12:12-14:30         |(2:18)|
|--------------------------------------------------------------|
\end{verbatim}



\subsection{Exemple}

\subsubsection{Utilizare de Bază (Modul \texttt{-l,-lb})}

Afișarea tuturor sesiunilor reușite (-l)
\begin{lstlisting}[language=bash]
sudo ./mylast.sh -l
\end{lstlisting}

Afișarea încercărilor de autentificare eșuate (-lb)
\begin{lstlisting}[language=bash]
sudo ./mylast.sh -lb
\end{lstlisting}

\subsubsection{Utilizare cu Filtre (Modul \texttt{-l})}

Limitarea numărului de intrări (-n)
\begin{lstlisting}[language=bash]
sudo ./mylast.sh -l -n 5
\end{lstlisting}

Filtrare după data de început (-s)
\begin{lstlisting}[language=bash]
sudo ./mylast.sh -l -s 2023-12-01-12-00
\end{lstlisting}

Filtrare după data de sfârșit (-t)
\begin{lstlisting}[language=bash]
sudo ./mylast.sh -l -t 2024-01-15-23-59
\end{lstlisting}

Verificarea prezenței la un moment dat (-p)
\begin{lstlisting}[language=bash]
sudo ./mylast.sh -l -p 2023-12-25-14-30
\end{lstlisting}

\section{Arhitectura Proiectului}

\subsection{Arhitectura Funcțiilor de Bază}

\subsubsection{Argumentul Fundamental al Funcțiilor}
Scriptul definește o variabilă \texttt{SYSLOG} (\texttt{/var/log}) ce stochează locația system log-urilor. În continuare, următoarele fișiere sunt concatenate într-un singur string, care este redirectat spre funcțiile de bază \texttt{mylast/mylastb}:

\begin{lstlisting}[language=bash]
LOG1="$SYSLOG/auth.log"
LOG2="$SYSLOG/auth.log.1"
LOG3="$SYSLOG/auth.log.2.gz"
LOG4="$SYSLOG/auth.log.3.gz"
LOG5="$SYSLOG/auth.log.4.gz"
\end{lstlisting}

\subsubsection{\texttt{mylast}}

Funcția \texttt{mylast} implementează o mașină de stări pentru a urmări sesiunile utilizatorilor.

\begin{enumerate}
    \item \textbf{Detectare Autentificare:} Scanează după ``new session''.
    \item \textbf{Deconectare/Eliminare:} Scanează după ``removed session'' sau ``logged out''.
    \item \textbf{Oprire Sistem:} Dacă este detectat ``system is powering down'', toate sesiunile active sunt închise.
\end{enumerate}

Această mașină de stări este construită pe baza a 2 variabile: \\ \\ \textbf{Queue} coada - salvează utilizatorii autentificați dar care încă nu au fost deconectați \\ \textbf{Session} sesiunea - salvează utilizatorii conectați și deconectați ulterior. \\ \\
Scriptul parcurge system log-urile linie după linie. Dacă observă un nou utilizator, atunci acesta este salvat în queue. Dacă acest utilizator este deconectat, atunci acesta este scos din queue și este salvat în session.

\subsubsection{\texttt{mylastb}}
Funcția \texttt{mylastb} procesează sesiunile în care utilizatorul a greșit parola. Funcționalitatea de bază este asigurată de \texttt{grep} după șirul de caractere: "password check failed".

\subsection{Arhitectura Funcțiilor de Filtrare}

\subsubsection{\texttt{-n, -s, -t, -p}}
Folosind totalitatea sesiunilor deja procesate de funcția \texttt{mylast}, verifică dacă data și timpul corespund cerințelor filtrului \texttt{(s, t, p)}, inclusiv și numărul de sesiuni ce necesită afișate \texttt{(n)}.

\subsection{Arhitectura Funcțiilor Ajutătoare}

\subsubsection{\texttt{get\_len}}
Determină numărul de linii într-un string delimitat prin spații. Folosește la parcurgerea variabilelor \textbf{Queue} și \textbf{Session}.

\subsubsection{\texttt{get\_time}}
Parsează linia de system log și returnează timpul în care s-a înregistrat linia respectivă de jurnal. Folosește la obținerea timpului de autentificare al utilizatorului.

\subsubsection{\texttt{get\_date}}
Parsează linia de system log și returnează data la care s-a înregistrat linia respectivă de jurnal. Folosește la obținerea datei de autentificare a utilizatorului

\subsubsection{\texttt{get\_duration}}
Calculează diferența dintre timpul de început ($T_s$) și timpul de sfârșit ($T_e$) al unei sesiuni. Folosește la formatarea finală a datelor despre timp pentru o sesiune încheiată a unui utilizator.

\end{document}
